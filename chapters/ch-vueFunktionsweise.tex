 \subsection{Funktionsweise/Anwendung}
\subsection*{Routing}
In Vue.js k\"onnen Seiten durch Routing im \ac{HTML} Dokument verlinkt werden. Die Seiten werden hierzu davor definiert.
Das kann die Reaktionsf\"ahigkeit der Webanwendung deutlich verbessern, da die Seite dynamisch auf dem Kontext der Seite aufgebaut ist, bedeutet, dass die Seiten einmal geladen werden und sonst nur angezeigt werden m\"ussen und die einzelnen Bereiche aktualisiert werden, wenn sich beispielsweise Daten \"andern. Dabei m\"ussen die verschiedenen Views oder Seiten unterschieden werden. Vue.js unterst\"utzt hierf\"ur eine Router Bibliothek, \textbf{vue-router}\cite{Zynda2017}.
Um die verschiedene Seiten zu definieren und zu verlinken, wird eine Instanz des Routers erstellt, in der ein oder mehrere Routen \"ubergeben werden. Die Definition eines Routers ist ein Array, das mehrere Routen enth\"alt, mit dem Attribut \texttt{path}, das die Verlinkung zur passenden Seite enth\"alt. Diese Instanziierung wird an die Vue-Instanz \"ubergeben.
Um die Implementierung zu vollenden, muss im \ac{HTML} Dokument die Routen aufgef\"uhrt werden um sie im Web wiederzugeben. Das wird wie folgt gemacht:  \texttt{<router-view></router-view>}\cite{Bemenderfer2017}.\\

\subsection*{Templating}
Templating ist eine Vordefinierung eines Designs oder eines Formats f\"ur ein Dokument. Dies kann f\"ur Vorlagen in Word oder \"ahnliches sein, aber auch f\"ur \ac{HTML} Dokumenten als Vorlage f\"ur das Design der Webseite und Funktionen von einzelnen Komponenten. Diese Vordefinierung ist universell und kann mit jeden beliebigen Daten gef\"ullt werden\cite{DictionaryTemplating}. Templating erlaubt, Werte bzw. Daten vom Model in die View zu binden. Seit Version 2.0 wird auch JavaScript Templating mit \ac{HTML} Templating in Vue.js unterst\"utzt. \\
Das meist benutzte Symbol ist die doppelte geschweifte Klammer. Durch diese Art von Templating wird eine One-Way-Bindung vom Model zum Template aufgebaut. Mit einer One-Way-Verbindung k\"onnen Daten von dem Model zum Template gesendet werden, aber nicht von dem Template zum Model\cite{AlligatorTemplating2016}.
%TODO Abbildung
Reaktive Datenbindung ist eine der Haupteigenschaften von Vue.js, sie speichert die Daten, wie Arrays oder JavaScript Variablen, in Verbindung mit dem \ac{HTML} Dokument. Die One-Way Bindung, wie die Abbildung \ref{fig:MVVMVue} zeigt, aktualisiert das \ac{HTML} Dokument automatisch, wenn in JavaScript die \texttt{msg} manipuliert wird.\\
 Um eine Two-Way Bindung zu erzeugen und \"uber zum Beispiel einem Input Feld die Daten zu \"andern, muss an das \texttt{input} Tag ein \textbf{v-model} integriert und an die Identifikation das zu \"andernden Wertes gebunden werden. In Abbildung \ref{fig:CodeVueInit} m\"usste unterhalb der Zeile \texttt{<h1> \{\{msg\}\} </h1>} ein Input Feld mit der Bindung angef\"ugt werden (\texttt{<input v-model=\grqq msg\grqq>})\cite{Gore2016}. Wird ein neuer Text in das Input Feld geschrieben, um somit den Wert zu \"andern, wird der Wert durch die reaktive Datenbindung automatisch und mit einer geringen Reaktionszeit aktualisiert.

\subsection*{Events}
Die Autoren Etzion und Niblett erkl\"aren in ihrem Buch \enquote{Event processing in action} ein Event folgenderma\ss{}en:
\begin{quotation}
\enquote{ An event is an occurrence within a particular system or domain; it is
something that has happened, or is contemplated as having happened in that
domain. The word event is also used to mean a programming entity that represents such an occurrence in a computing system\cite{Etzion&Niblett2011}.}
\end{quotation}
Das bedeutet, dass ein Event auftritt, wenn etwas passiert, wie beispielsweise ein Mausklick bzw. das dr\"ucken auf einen Bildschirms oder das Scrollen.
Um ein Event in unser \ac{HTML} Dokument an  wie in etwa ein Button zum Klicken zu binden, wird in Vue.js ein \texttt{v-on} verwendet. Hierbei k\"onnen auf verschiedene Art und Weise ein Button oder Methoden, die ausgef\"uhrt werden sollen, gebunden werden.\\
F\"ur einen Button ist die \texttt{click}-Methode die standardgem\"a\ss{}e Weise des Events. F\"ur das folgende Beispiel wird ein Button zum hochz\"ahlen einer Zahl verwendet: \texttt{<button v-on:click=\grqq counter += 1\grqq>{{counter}}</button>}\cite{VueDokumentationEvent2018}. Das Klick-Event wird an den Button gebunden, sodass die Variable \texttt{counter} beim Eintreten des Events ver"andert wird.\\
Statt das Event an ein Objekt zu h\"angen, kann \texttt{v-on} auch an Methoden gebunden werden, um eventuell komplexere Ausf\"uhrungen durchzuf\"uhren. Dabei  kann ebenfalls eine \texttt{click}-Methode verwendet werden, die auf den Namen der Methode, die ausgef\"uhrt werden soll, hinweist (\texttt{v-on: click=\grqq Methodennamen\grqq})\cite{VueDokumentationEvent2018}. Zus\"atzlich k\"onnen Parameter in dem Methodennamen angegeben werden, die in der Methode interpretiert werden k\"onnen. 
Ebenso werden sogenannte \textit{Modifier} von Vue.js unterst\"utzt, um die immer wiederkehrenden Aufrufe w\"ahrend den Events handzuhaben.
Modifiers sind Schl\"usselw\"orter, die den Grad des Zugriffsrechts und die Sichtbarkeit auf Variablen, Funktionen oder Klassen regeln.
Ein Beispiel w\"are das \texttt{preventDefault}, das typischerweise aufgerufen wird, wenn das standardgem\"ase Verhalten des Browsers verhindert werden soll \cite{VueDokumentationEvent2018}.

\subsection*{Validation}
Eine Validierung durchzuf\"uhren, ist vor allem bei Formularen und Registrierungen wichtig. Die Validierung \"uberpr\"uft die Richtigkeit der Daten und die Erf\"ullung der gegebenen Anforderungen. Browser besitzten \"ublicherweise nativ die Validierung einer Form, da jedoch jeder Browser Objekte unterschiedlich handhaben k\"onnen, ist die Vue.js basierte Validierung eine gute L\"osung, um einheitlich zu bleiben. Zu Beginn sollte in dem \ac{HTML} Dokument ein \texttt{form} Tag vorhanden sein, ?indem die Felder zur interaktiven Anwendung eines Formulars hinzugef\"ugt werden?. Durch das schon bekannte \texttt{v-model} k\"onnen Bedingungen an ein Attribut gebunden werden und in JavaScript anhand dessen verarbeitet werden k\"onnen.
Standardgem\"a\ss{} kann bei Eingabe einer Zahl, das \"uber das \texttt{type} Attribut \"uberpr\"uft werden kann, ein Minimum und Maximum angegeben, das dann in Vue.js mit einem \texttt{v-if} \"uberpr\"uft wird und die passende Nachricht an den Anwender weiter gegeben werden kann. Diese Validierung erfolgt haupts\"achlich in JavaScript, Clientseitig oder Serverseitig. Dabei gibt es Plugins, wie Vee-Validate und Vuelidate, die schon bestimmte Regeln mit sich bringen und das Validieren in Vue.js vereinfachen\cite{VueDokumentationValidierung2018}.\\
Mit Vee-Validate wird Clientseitig validiert und durch das Hinzuf\"ugen von einem \texttt{v-validate} Attribute, das auf das Model, das \"uberpr\"uft werden soll, weist. Vee-Validate bringt eigene Regeln mit, die mit \texttt{data-vv-rules} auf das jeweilige Model angepasst werden kann\cite{BemenderferVeeValidate2017}.\\
Vuelidate ist eine Modelbasierte Validierung, was eine flexiblere, auf das Minimum  reduzierte M\"oglichkeit der Validierung ist. In dem von Vuelidate eigenen \$v Schema werden die Validierungsm\"oglichkeiten gespeichert (\texttt{this.\$v[propertyName]}). Das kann in dem \ac{HTML} Dokument durch \texttt{v-if=\grqq !\$v.emailValue.required\grqq} auf Vorhandensein \"uberpr\"uft werden. Ebenso kann eine eigene Validierung erstellt werden, wenn die gew\"unschte \"Uberpr\"ufung nicht vorhanden ist, indem  sie als Funktion dargestellt wird\cite{BemenderferVuelidate2017}.

\subsection*{Komponente}
Komponenten sind im Allgemeinen gesprochen Bereiche bzw. Teile eines Systems, die zusammen arbeiten k\"onnen.
In Vue.js helfen Komponenten, das standard-\ac{HTML} Dokument zu erweitern. 
Um ein Komponent zu erstellen, muss dies erstmals mit \texttt{Vue.component(tag, constructor)} registriert werden, um die Komponente nutzen zu k\"onnen\cite{VueDokumentationComponents2018}. In dem Konstruktor wird die Funktion definiert, die beim Verwenden des Komponenten ausgef\"uhrt wird, bzw. die Optionen mit den beinhaltenden Daten f\"ur das \ac{HTML} Dokument. Eine Option, die vorhanden sein muss, ist die option \texttt{data}. \texttt{data} sollte f\"ur die Wiederverwendbarkeit eine Funktion sein, die das unabh\"angige Objekt zur\"uck geben kann. Ist data keine Funktion, wird f\"ur jedes seperat erstellte Komponent das gleiche ausgef\"uhrt. Die Wiederverwendung kann per Name, der als \texttt{tag} \"ubergeben wird, definiert werden. Hei\ss{}t, wenn eine eigene Button-Komponente erstellt wird, kann man sie mehrmals verwenden, \textsc{Beispiel} die jedoch jeweils f\"ur sich ein eigener Button ist und die Funktion, die hinter der Komponente steht, wird f\"ur jeden Button seperat ausgef\"uhrt, da die Instanz jedes mal aufs neue erstellt wird\cite{VueDokumentationComponents2018}.