\subsection{Andere Frameworks}
Analog zu Vue.js gibt es weitere Bibliotheken bzw. Frameworks, die die Anforderungen ebenfalls als Reactive Programming betreiben und mit dem Entwurfsmuster \ac{MVVM} arbeiten. Ein Beispiel w\"are \textbf{React.js}.
React.js ist ein von Facebook erstellte JavaScript Bibliothek, das dem Entwickler beim Erstellen von Oberfl\"achen hilft. Konzerne wie Whatsapp, Instagram und Facebook nutzen die Frontend Bibliothek. Das Ziel, das React.js verfolgt, ist es einfacheren Code zu schreiben, um die einzelnen Bestandteile besser zu verstehen und weniger komplex zu halten. Die wesentliche Bestandteile von React.js sind die Komponentenarchitektur, der virtuelle \ac{DOM} und die Browserkompatibilit\"at.\\
React.js Komponente sind \"aquivalent zu Web Komponenten, die mit \texttt{React.createClass} erstellt werden. Innerhalb gibt es Funktionen wie \texttt{render}, die das \ac{HTML} Dokument f\"ur das Web pr\"asentierbar machen. Dazu k\"onnen eigene Funktionen definiert werden. Bei React.js wird die standard \texttt{onClick} Methode als Attribut auf den Button gesetzt und durch geschweifte Klammer kann auf die Funktionen verwiesen werden (\texttt{<button onClick={this.add}>}). Ein besonderes Attribut von React.js ist das \texttt{State}. Dieses Attribut beinhaltet die zu ver\"andernden Daten und kann die Aktualisierung im \ac{HTML} Dokument durchf\"uren. Somit muss auf die Anpassung des \ac{DOM} keine R\"ucksicht genommen werden. Die Komponente werden in React.js innerhalb der \texttt{render} Methode definiert, da React.js JSX eine schlanke Syntaxerweiterung zum Schreiben von Markups verwendet. \"Anderungen am Stil der Seite oder des Buttons wird innerhalb der \texttt{createClass}, wo auch die \texttt{render} Methode f\"ur die Definition des \ac{HTML} Dokuments implementiert wird, erstellt (\texttt{return {backgroundColor: \#fff};}). Hierbei wird erkannt, dass die \"ublich bekannte Trennung der Bereiche \ac{HTML}, \ac{CSS} und JavaScript nicht stattfindet\cite{Kogel2015}. \\
Die Bearbeitung durch JSX in React.js wird in der Regel nicht direkt in dem \ac{DOM} des Browsers stattfinden, sondern mit einem virtuellen \ac{DOM}, das ein JavaScript Objekt ist, das zur Bearbeitung genutzt wird. Bei einer Ver\"anderung wird jeweils ein neues Objekt, also ein neuer virtueller \ac{DOM}, erstellt. Dabei wird der virtuelle \ac{DOM} mit dem Browser \ac{DOM} verglichen und aufgelistet. Die \"Anderungen werden erst zum Zeitpunkt des Batch an den Browser geschickt und aktualisiert\cite{Skirzynski2015}. Im Vergleich zu Vue.js ist React.js einer der \"ahnlichsten Bibltiotheken. Beiden nutzen einen virtuellen \ac{DOM}, eine reaktive, zusammensetzbare Benutzeroberfl\"ache und nutzen durch das Routing und des States um die \"Anderungen zu fokusieren. Die folgende Auflistung zeigt die Unterschiede in Vue.js und React.js in den Positionen der Leistung, im Templateing und JSX und die Skalierbarkeit des Systems.
\begin{itemize}
\item \textbf{Leistung:} Vue.js und React.js sind in Hinsicher der Leistung \"ahnlich schnell, was f\"ur die Entscheidung irrelevant ist. Die genauere Leistung wird im sp\"ateren Teil genauer bezeichnet\cite{VueDokumentationOther2018}.
\item \textbf{Templating} Im Gegensatz zu Vue.js ist in React.js \ac{HTML} und \ac{CSS} zusammen in JavaScript mit Hilfe von JSX geschrieben, das mit Hilfe einer \texttt{render} Funktion, die ebenfalls in Vue.js vorhanden ist, an der Benutzeroberfl\"ache daregestellt werden kann. Ebenso werden Werkzeuge wie Typen\"uberpr\"ufung in React.js besser unterst\"utzt\cite{VueDokumentationOther2018}. Die konkrete Teilung zwischen JavaScript, \ac{HTML} und \ac{CSS} ist in Vue.js deutlich erkennbar und f\"ur die Meisten Entwickler \"ubersichtlicher\cite{Haupt2018}. In Vue.js ist das Templating f\"ur das Rendering des \ac{HTML} Dokuments. F\"ur viele Entwickler, die in \ac{HTML} Erfahrung haben, wird es einfacher sein, den Code zu lesen und zu interpretieren\cite{VueDokumentationOther2018}.
\item \textbf{Skalierbarkeit:} F\"ur die Skalierung f\"ur gr\"o\ss{}ere Anwendungen gibt es in Vue.js sowie in React.js viele Routing L\"osungen. In React.js gibt es die Frameworks Fluex oder Redux und in Vue.js das Vuex, auch Redux kann in Vue.js integriert werden\cite{Peters2017}. Vue.js beinhaltet einen \ac{CLI}, das die Einbindung neuer Projekte mit Werkzeugen zum Bauen des Projekts wie webpack oder Browserify erm\"oglicht.
Vor allem in React.js m\"ussen die Build Systeme, zum Bauen des Projekts sich zuerst angeeignet werden, was in Vue.js beispielsweise das Webpack \"ubernimmt\cite{VueDokumentationOther2018}.
\end{itemize}
Ein weiteres Framework, das f\"ur die Frontend Entwicklung hilfreich ist, ist \textbf{Angular.js}. Angular.js ist den Meisten bekannter als Vue.js oder React.js. Viele Komponenten oder Syntaxen von Angular.js waren f\"ur Vue.js Inspirationen, weshalb es vielen Angular.js Entwicklern einfacher f\"allt sich in Vue.js einzuarbeiten. Angular.js ist deutlich komplexer und hat mehr Vorgabe in Hinsicht der Softwarearchitektur als Vue.js. Ebenso ist die Leistung von Vue.js deutlich besser und leichter zu optimieren. 
Die Entscheidung f\"ur Vue.js basierte auf der immer gr\"o\ss{}erer werdenden Popularit\"at seit 2016, was durch die Google Trends Statistik deutlich hervorgeht\cite{GoogleTrends2018}.  F\"ur die Gr\"o\ss{}e des zu implementierende Projekt ist Angular.js zu \"uberladen. Vue.js oder React ist kompakt genug, f\"u eine Webanwendung zur Darstellung von Positionsdaten jedoch ausreichend. Dabei ist die Leistung sehr entscheidend. \newline
in einer Studie einer Bachelorarbeit aus Schweden wurde der Performancevergleich von Angular 2, Aurelia,  Ember, Vue und weiteren getestet. Dabei wurden die Befehle \texttt{Create}, \texttt{Delete} und \texttt{Update} mit jeweils 1000 Reihen getestet.\cite{Svensson2015}
\begin{itemize}
\item \textbf{Angular.js 2:} Im Ganzen hatte Angular 2 in jedem Testszenario die beste Leistung, dabei hat Angular 2 eine deutliche Verbesserung zu seinem Vorg\"anger 1.5\cite{Svensson2015}.
\item \textbf{Aurelia:} Aurelia hat in dem Szenario das zweitbeste Ergebnis erziehlt, jedoch schnitt Aurelia bei dem Befehl Update mit unter Anderem als schlechtestestes Frameworks ab\cite{Svensson2015}.
\item \textbf{Ember:} Eins der schlechtesten Frameworks ist Ember. Nur in dem Befehl Delete erreichte Ember einen durchschnittliches Ergebnis\cite{Svensson2015}.
\item \textbf{Vue.js:} Vue.js war eins der schnellsten Frameworks, vor allem bei den Befehlen Create und Update\cite{Svensson2015}.
\end{itemize}

