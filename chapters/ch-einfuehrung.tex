\chapter{Einf\"uhrung}
Dieses Kapitel dient dazu, die Arbeit in ihrem thematischen Umfeld einordnen zu k\"onnen. Dazu wird zun\"achst ein Einblick in das Aufkommen der ersten Ideen, die Entwicklung und dem aktuellen Stand des autonomen Fahrens gegeben. Anschlie\ss{}end wird auf den Inhalt des Projektes und den Aufbau der Arbeit eingegangen.
\section{Motivation}
Moderne Fahrerassistenzsysteme unterst\"utzen heutzutage den Fahrer in jeder erdenklichen Situation. Ob die Einhaltung der Fahrspur, Abstandsregler zum vorderen Auto oder die Einparkhilfe, alle Assistenzsysteme tragen zum autonomen Fahren bei. Durch die immer gr\"o\ss{}er werdende Digitalisierung\cite{Hesse2015}, werden auch die
Assistenzsysteme immer weiter optimiert bis hin zum fahrerlosen Fahren, das bis 2025 realisierbar sein wird\cite{Dahlmann2018}.
\section{Ziel der Arbeit}
Durch vorhergehende Studentenprojekte wurden Hot Wheels A.I. Modellautos im Ma\ss{}stab von 1 zu 32 mit eigene Hardware und Firmware modifiziert. Diese k\"onnen sich bereits auf der dazugeh\"origen Fahrbahn anhand eines Graustufenverlaufs autonom bewegen. Zus\"atzlich ist eine manuelle Ansteuerung \"uber Bluetooth m\"oglich.
Zu entwickelndes Gesamtsystem. \par
Im Rahmen der Bachelorarbeit soll ein System entwickelt werden, welches mit Hilfe einer Deckenkamera die im Grundsystem \ac{C.A.M.} vorgestellten Autos erfasst und \"uber eine individuell zusammengestellte Fahrbahn visuell darstellt. Dazu wird das Gesamtsystem in zwei Teilsysteme aufgeteilt. Jedes Teilsystem wird jeweils von einem Bacheloranten bearbeitet. Dabei sind die Verantwortlichkeiten f\"ur die Systemaufgaben entsprechend aufgeteilt. Da das System gemeinsam entwickelt wird, sind Teile dieser Arbeit gemeinschaftlich erarbeitet und von anderen Gruppenmitgliedern niedergeschrieben worden. Die Diagramme in Kapitel \ref{sec:AnfAnalyse} f\"ur das Gesamtsystem wurde von Felix Grammling erstellt.
\subsection*{Teilsystem}
Das Teilsystem der vorliegenden Bachelorarbeit umfasst die Visualisierung einer individuell zusammengestellten Fahrbahn mit Position von autonom fahrenden Modellautos und zugeh\"orige Live Daten des getrackten Fahrzeugs. Die Informationen, wie zum Beispiel Koordinationspunkte der Fahrbahn oder die aktuelle Position der Fahrzeuge,
werden \"uber das erste Teilsystem ermittelt und zur Verf\"ugung gestellt. Da die Aktualisierung der getrackten Position des Fahrzeuges in einem abgestimmten Zyklus geschieht, soll die Visualisierung optimiert werden und die Bewegungen des Fahrzeuges so reibungslos wie m\"oglich sein. Es werden Algorithmen zur Fahrspurvisualisierung und neue Technologietrends und Werkzeuge wie Vue.js genutzt.\par \smallskip 
Des Weiteren soll eine Bluetooth Verbindung zu den Fahrzeugen hergestellt werden, um vorerst manuelle \"Anderungen wie zum Beispiel an der Geschwindigkeit vorzunehmen. Optional sollte die Visualisierung mehrere Fahrzeuge und Hindernisse auf der Fahrbahn darstellen.
\subsection*{Aufbau der Bachelorarbeit}
In Kapitel \ref{sec:Vue} wird auf das Framework Vue.js eingegangen, mit dem die Webanwendung erstellt wird und das dazugeh\"orige Entwurfsmuster \ac{MVVM} in Kapitel \ref{sec:MVVM} beschrieben.\par\smallskip 
Kapitel \ref{sec:req} werden die  Anforderungen an das Gesamtsystem erl\"autert. Zu den funktionalen und nicht funktionalen Anforderungen an das Teilsystem wird in Kapitel \ref{sec:TeilVisu} eingegangen. \par\smallskip 
Die Anforderungsanalyse in Kapitel \ref{sec:AnfAnalyse} beinhaltet das Gesamtsystem mit Kontextdiagramm und die dazugeh\"origen Anwendungsf\"alle von Kapitel \ref{sec:req} in Kurz- und Langbeschreibung.\par \smallskip
Kapitel \ref{sec:Systementwurf} befasst sich mit der technischen Umsetzung des Teilsystems, dabei werden Mockups und die Architektur beschrieben. \par \smallskip
Das Implementierungskapitel \ref{sec:Impl} gibt die Weise der Realisierung der Webanwendung wieder sowie Codebeispiele zur Verdeutlichung der verwendeten Technologien.\par \smallskip
Zum Schluss in Kapitel \ref{sec:schluss} wird das Ergebnis und die Erfahrung dargelegt und bietet die darauf folgende Schl\"usse an.
