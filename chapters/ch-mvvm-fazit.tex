\subsection{Fazit}
Bevor Software-Projekte mit Architekturmuster realisiert worden waren, wurde die grafische Oberfl\"ache in die Mitte der Anwendung implementiert. Dieser einfache Einsatz wird Smart UI anti-pattern genannt, das wie jedes Muster seine Vorteile hat, aber bei gr\"o\ss{}eren Projekten zu einer H\"urde werden kann. Das Gestalten der verschiedenen Architekturmustern bew\"altigen die Probleme des Smart UI anti-pattern. Jedes Architketurmuster ist \"ahnlich aufgebaut, jedes trennt die Benutzerschnittstelle mit den Daten und deren Verarbeitung. Die Kommunikation zwischen den drei Komponenten h\"angt von von der grundlegenden Umgebung ab. Beispeilsweise ist die Synchonisation mehrerer Views durch einen \enquote{observer}, einen Beobachter, der die \"Anderungen weitergibt und somit die View aktualisiert, n\"utzlich, aber in vielen Situationen nicht zug\"anglich. Somit sind auch andere Mechanismen brauchbar, wie die Datenbindung die in dem Architekturmuster \ac{MVVM} zwischen dem ViewModel und der View. Die Anwendung des Mechanismus zur Kommunikation h\"angt von der Anwendung selbst, der Programmiersprache, der Frameworks, des angewendeten Musters und der pers\"onlichen Preferenz aufgezeigt\cite{Bragge2013}. In der folgenden Tabelle\cite{Syromiatnikov2014} werden die Vorteile und Nachteile des jeweiligen Entwurfsmusters. Dies zeigt, dass kein Gewinner ausgew\"ahlt werden kann, jeder hat seine eigenen individuellen Vorz\"uge.
\begin{table}[h]
\centering
\caption{Vorteile und Nachteile der Architekturmuster}
\label{ProConsPattern}
\begin{tabular}{p{2cm}p{7cm}p{7cm}}
\toprule
Muster & Vorteile & Nachteile \\ \midrule
\ac{MVVM}              
& Unterst\"utzt mehrere Views f\"ur das gleiche Model und aktualisiert View und ViewModel automatisch
& 
Das Benutzten von \enquote{Beobachter} schw\"acht die Leistung. Beruht auf der zugrunde liegenden Technologie.  \\ \hline
\ac{MVC}     
 & 
Gut geeignet f\"ur Web Anwendungen, da View und Controller hier getrennt gehandhabt wird.
 &
 Die Abh\"angigkeit von View und Controller, was hier nicht gegeben ist, sind in manchen Steuerelementen von n\"oten.  \\ \hline
\ac{MVP}            
&
Flexibilit\"at und Freiheit zum Verwenden der verschiedenen Mechanismen und kann an viele Anwendungsszenarien verwendet werden. 
&
Keine strikte Trennung der Komponenten, das bei komplexere Codestellen zu Problemen f\"uhren kann.  
\\ \bottomrule
\end{tabular}
\end{table}
