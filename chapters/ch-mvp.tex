\setcounter{secnumdepth}{3}
\subsubsection{Model-View Presenter}
Das Architekturmuster \ac{MVP} basiert auf dem \ac{MVC} Muster. Die Komponente wurden anhand hohen Flexibilit\"at angepasst und die M\"angel des vorherigen Musters (\ac{MVC}) werden besser gehandhabt.  Das Hauptmerkmal des \ac{MVP} Musters ist der Presenter, das direkten Zugriff auf die View und das Model besitzt und deren Zusammenspiel handhabt. Die View und das Model k\"onnen kleine Situationen selbst meistern, was das ganze die Komplexit\"at des Presenters reduziert, da dieser nur noch f\"ur die komplexen Anforderungen verantwortlich ist. Somit ist das Architektumuster \ac{MVP} ist das flexibelste der MV* Familie, da es dem Entwickler viel Freiraum und Kontrolle gibt. Zum Beispiel kann die Datenbindung da genutzt werden, wo es dem Entwickler am besten passt\cite{Syromiatnikov2014}.
Die Views sind wie auch beim \ac{MVC} Muster f\"ur die Anschauung der Daten zust\"andig und das Model f\"ur die Datenhaltung verantwortlich. Martin Fowler verglich 2006 das Architekturmuster \ac{MVC} mit \ac{MVP} so, dass der Controller in \ac{MVC} bei  \ac{MVP} ein Teil der View ist\cite{Bragge2013}. Die Reaktion auf die Aktivit\"aten des Users ist im Presenter enthalten. Er kann entscheiden wie das Model manipuliert und ver\"andert werden kann, hei\ss{}t er \"ubernimmt bzw. integriert die Rolle des Application Model, das man von \ac{MVC} kennt. Zusammenfassend zu sagen ist, dass der Presenter die Business Logik zur Verf\"ugung stellt.
In der Regel verh\"alt sich die Kommunikation gleich wie beim \ac{MVC}. Durch Aktionen des Anwenders, wie eine Mausbewegung, werden Interactor Events ausgel\"ost. \textbf{?} Diese Interactionen werden vom Presenter interpretiert. Die Models werden erstellt sowie Kommandos und die Views\cite{Potel1996}.




