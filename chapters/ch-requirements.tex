\chapter{Requirements}
Requirements, in deutsch \enquote{Anforderungen}, beschreiben die Ziele eines Kunden, die angeforderten Funktionalit\"aten und Eigenschaften des Systems, sowie in welcher Qualit\"at dies umgesetzt werden muss\cite{Heini2010}. Die Beschreibung, das Pr\"ufen, das Verwalten sowie die Analyse solcher Requirements wird als des Requirements Engineering beschrieben\cite{Sophist2011}.

\section{Funktionale Requirements}
Funktionale Requirements charakterisieren die Leistungen an das System von Anwendungsf\"allen. Zus\"atzlich definieren die funktionalen Requirements die technischen Funktionen eines Systems, die als L\"osung f\"ur die Problematik der Anwendungsf\"alle fungieren\cite{Goll2011}.

\subsection*{Funktionales Top Level Requirement}
\textbf{Requirement 1000:}\\
Das System muss Anwendern eine Fahrbahn visualisieren und die darauf vorhandene Position eines Fahrzeugs aufzeigen.
\subsection*{Requirements an die Anwendungsf\"alle}
\begin{itemize}
\item \textbf{Requirement 1110:}\\
Der Anwender soll das Fahrzeug vom System aus ein- und ausschalten k\"onnen.
\item \textbf{Requirement 1120:}\\
Der Anwender soll durch Eingabe einer Zahl die Geschwindigkeit des Fahrzeugs anpassen.\\
\end{itemize}

\section{Nicht funktionale Requirements}
Nicht funktionale Requirements definieren Forderungen an den L\"osungsbereich wie die Architektur, Technologien oder an die Qualit\"at wie zum Beispiel die Performance. Beispiele hierf\"ur sind sie Bedienbarkeit, das eingesetzte Betriebssystem, die verwendete Hardware und die genutzte Programmiersprache\cite{Goll2011}.

\subsection{Anforderungen an den Nutzungskontext}
\begin{itemize}
\item \textbf{Requirement 1210:}\\
Das System soll \"uber eine Webanwendung verf\"ugbar sein.
\end{itemize}
\textbf{Darstellung:}
\begin{itemize}
\item \textbf{Requirement 1220:}\\
Durch Intervall\"anderungen der Positionen, sollen die \"Anderungen zeitlich schnell aktualisiert werden.
\item \textbf{Requirement 1230:}\\
\textsc{Optional:} Die Darstellung des Fahrzeugs in der Webanwendung sollte in einer reibungslose Bewegung dargestellt werden
\end{itemize}
\subsection{Anforderungen an die Implementierung}
\textbf{Architektur:}
\begin{itemize}
\item \textbf{Requirement 1310:}\\
Die Anwendung zeigt eine \ac{SPA}-Architektur auf.\\ SICHER?
\end{itemize}
\textbf{Technologien:}
\begin{itemize}
\item \textbf{Requirement 1320:}\\
Das System soll mit der Frontend Bibliothek Vue.js umgesetzt werden.
\item \textbf{Requirement 1330:}\\
Die Positionsdaten sollen per Webservices, wie Websockets \"ubermittelt werden.
\item \textbf{Requirement 1340:}\\
Die Verbindung zu den Fahrzeugen soll durch Web-Bluetooth verbunden werden.
\item \textbf{Requirement 1350:}\\
\textsc{Optional:} Durch die gegebene Zeit und der Strecke soll die Geschwindigkeit durch Berechnungen dargestellt.
 werden.
\end{itemize}