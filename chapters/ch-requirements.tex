%\chapter{Requirements}
%Requirements, in deutsch \enquote{Anforderungen}, beschreiben die Ziele eines Kunden, die angeforderten Funktionalit\"aten und Eigenschaften des Systems, sowie in welcher Qualit\"at dies umgesetzt werden muss\cite{Heini2010}. Die Beschreibung, das Pr\"ufen, das Verwalten sowie die Analyse solcher Requirements wird als des Requirements Engineering beschrieben\cite{Sophist2011}.
%
%\section{Funktionale Requirements}
%Funktionale Requirements charakterisieren die Leistungen an das System von Anwendungsf\"allen. Zus\"atzlich definieren die funktionalen Requirements die technischen Funktionen eines Systems, die als L\"osung f\"ur die Problematik der Anwendungsf\"alle fungieren\cite{Goll2011}.
%
%\subsection*{Funktionales Top Level Requirement}
\textbf{Requirement 1000:}\\
Das System muss Anwendern eine Fahrbahn visualisieren und die darauf vorhandene Position eines Fahrzeugs aufzeigen.
\subsection*{Requirements an die Anwendungsf\"alle}
\begin{itemize}
\item \textbf{Requirement 1110:}\\
Der Anwender soll das Fahrzeug vom System aus ein- und ausschalten k\"onnen.
\item \textbf{Requirement 1120:}\\
Der Anwender soll durch Eingabe einer Zahl die Geschwindigkeit des Fahrzeugs anpassen.\\
\end{itemize}
%
%\section{Nicht funktionale Requirements}
%Nicht funktionale Requirements definieren Forderungen an den L\"osungsbereich wie die Architektur, Technologien oder an die Qualit\"at wie zum Beispiel die Performance. Beispiele hierf\"ur sind sie Bedienbarkeit, das eingesetzte Betriebssystem, die verwendete Hardware und die genutzte Programmiersprache\cite{Goll2011}.
%
%\subsection{Anforderungen an den Nutzungskontext}
\begin{itemize}
\item \textbf{Requirement 1210:}\\
Das System soll \"uber eine Webanwendung verf\"ugbar sein.
\end{itemize}
\textbf{Darstellung:}
\begin{itemize}
\item \textbf{Requirement 1220:}\\
Durch Intervall\"anderungen der Positionen, sollen die \"Anderungen zeitlich schnell aktualisiert werden.
\item \textbf{Requirement 1230:}\\
\textsc{Optional:} Die Darstellung des Fahrzeugs in der Webanwendung sollte in einer reibungslose Bewegung dargestellt werden
\end{itemize}
\subsection{Anforderungen an die Implementierung}
\textbf{Architektur:}
\begin{itemize}
\item \textbf{Requirement 1310:}\\
Die Anwendung zeigt eine \ac{SPA}-Architektur auf.\\ SICHER?
\end{itemize}
\textbf{Technologien:}
\begin{itemize}
\item \textbf{Requirement 1320:}\\
Das System soll mit der Frontend Bibliothek Vue.js umgesetzt werden.
\item \textbf{Requirement 1330:}\\
Die Positionsdaten sollen per Webservices, wie Websockets \"ubermittelt werden.
\item \textbf{Requirement 1340:}\\
Die Verbindung zu den Fahrzeugen soll durch Web-Bluetooth verbunden werden.
\item \textbf{Requirement 1350:}\\
\textsc{Optional:} Durch die gegebene Zeit und der Strecke soll die Geschwindigkeit durch Berechnungen dargestellt.
 werden.
\end{itemize}
\section{Requirements}

\subsection{Top-Level Requirement}
\begin{itemize}
\item \textbf{Top Requirement 1000:}\\
Das System C.A.M. soll die Lage eines Fahrzeuges \"uber eine Deckenkamera auf der Fahrbahn ermitteln und zur Visualisierung bereitstellen.
\end{itemize}

\subsection{Geforderte Anwendungf\"alle}

\begin{itemize}
\item \textbf{Anwendungsfall Tracking:}\\
Das System soll das Fahrzeug auf der Fahrbahn erfassen.
\item \textbf{Anwendungsfall Darstellung:}\\
Das System soll die Fahrzeugposition auf der Fahrbahn visuell darstellen.
\end{itemize}

\textbf{Anforderungen an den Anwendungsfall "Tracking"}
\begin{itemize}
\item  \textbf{Requirement 2100:}\\
Das System muss das Fahrzeug erkennen.
\item \textbf{Requirement 2200:}\\
Das System muss die Position des Fahrzeugs erkennen.
\item \textbf{ Requirement 2300:}\\
Das System muss die Richtung des Fahrzeugs erkennen.
\item\textbf{ Requirement 2400:}\\
Das System muss die Geschwindigkeit des Fahrzeugs erkennen.
\item \textbf{Requirement 2500:}\\
Das System muss die ermittelten Daten \"uber eine Schnittstelle ausliefern.
\end{itemize}

\textbf{Anforderungen an den Anwendungsfall "Darstellung"}
\begin{itemize}
\item \textbf{Requirement 3100:}\\
Das System muss als Webanwendung implementiert werden.
\item \textbf{Requirement 3200:}\\
Das System muss die Fahrbahn visualisieren
\item \textbf{Requirement 3300:}\\
Das System muss anhand der \"ubermittelten Daten das Fahrzeug auf der Fahrbahn visualisieren
\end{itemize}

\subsection{Anforderungen an die technischen Funktion}
\textbf{Anforderungen an Start-up und Shut-down}
\begin{itemize}
\item \textbf{Requirement 1100:}\\
Nach dem Starten des Systems m\"ussen alle Funktionen zur Verf\"ugung stehen.
\end{itemize}

\textbf{Anforderungen an die Fehlererkennung, -behandlung und -ausgabe}
\begin{itemize}
\item \textbf{Requirement 1200:}\\
Alle Fehler die w\"ahrend des Betriebs des Systems entstehen, sollten protokolliert und
gespeichert werden.
\item \textbf{Requirement 1210:}\\
Geht das System in einen undefinierten, unsicheren Zustand \"uber, sollte es automatisch in einen sicheren Zustand gebracht werden.
\end{itemize}

\textbf{Anforderungen an die Kommunikation}
\begin{itemize}
\item \textbf{Requirement 1300:}\\
Es sollen die ermittelten Fahrzeuginformationen von dem Tracking zu der Darstellung mittels eines Websockets \"ubertragen werden.
\item \textbf{Requirement 1400:}\\
Die externe Deckenkamera soll die Bilder \"uber eine USB-Schnittstelle an das System\"ubertragen.
\end{itemize}

\textbf{Anforderungen an die Qualit\"aten}
\begin{itemize}
\item \textbf{Requirement 1520:}\\
Das System soll im Bereich Mehrfachdetektion von Fahrzeugen sowie von Hindernissen erweiterbar sein.
\item \textbf{Requirement 1530:}\\
Das System soll die Mehrfachdetektion von Fahrzeugen sowie von Hindernissen visuell erweiterbar sein
\item \textbf{Requirement 1540:}\\
Das System soll die Fahrzeuge steuern k\"onnen
\end{itemize}